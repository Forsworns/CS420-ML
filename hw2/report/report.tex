\documentclass[UTF8]{article}

\usepackage{ctex}

\usepackage{amsmath}

\usepackage{amssymb}

\usepackage{amsthm}

\providecommand{\abs}[1]{\lvert#1\rvert}

\providecommand{\norm}[1]{\lVert#1\rVert}



\newtheorem{thm}{Theorem}

\newtheorem{lemma}[thm]{Lemma}

\newtheorem{fact}[thm]{Fact}

\newtheorem{cor}[thm]{Corollary}

\newtheorem{eg}{Example}

\newtheorem{ex}{Exercise}

\newtheorem{defi}{Definition}

\newtheorem{hw}{Problem}

\newenvironment{sol}

  {\par\vspace{3mm}\noindent{\it Solution}.}

  {\qed}



\usepackage{algorithm,algorithmic,bm,color}

\floatstyle{plain}

\newfloat{myalgo}{tbhp}{mya}



\newenvironment{Algorithm}[2][tbh]

{\begin{myalgo}[#1]\centering\begin{minipage}{#2}\begin{algorithm}[H]}%

{\end{algorithm}\end{minipage}\end{myalgo}}



% Use the postscript times font!

\usepackage{times}

\usepackage{soul}

\usepackage{url}

\usepackage[hidelinks]{hyperref}

\usepackage[utf8]{inputenc}

\usepackage[small]{caption}

\usepackage{graphicx}

\usepackage{amsmath}

\usepackage{booktabs}

\usepackage{algorithm}

\usepackage{algorithmic}\usepackage{amsfonts}

\urlstyle{same}



\makeatletter

\newenvironment{breakablealgorithm}

{% \begin{breakablealgorithm}

	\begin{center}

		\refstepcounter{algorithm}% New algorithm

		\hrule height.8pt depth0pt \kern2pt% \@fs@pre for \@fs@ruled

		\renewcommand{\caption}[2][\relax]{% Make a new \caption

			{\raggedright\textbf{\ALG@name~\thealgorithm} ##2\par}%

			\ifx\relax##1\relax % #1 is \relax

			\addcontentsline{loa}{algorithm}{\protect\numberline{\thealgorithm}##2}%

			\else % #1 is not \relax

			\addcontentsline{loa}{algorithm}{\protect\numberline{\thealgorithm}##1}%

			\fi

			\kern2pt\hrule\kern2pt

		}

	}{% \end{breakablealgorithm}

		\kern2pt\hrule\relax% \@fs@post for \@fs@ruled

	\end{center}

}

\makeatother



\setlength{\parindent}{0pt}

%\setlength{\parskip}{2ex}

\newenvironment{proofof}[1]{\bigskip\noindent{\itshape #1. }}{\hfill$\Box$\medskip}



\usepackage{enumerate,fullpage,proof}



% the following package is optional:

%\usepackage{latexsym} 

\usepackage{amsfonts}

\usepackage{subfigure}

\usepackage{booktabs}

\usepackage{stfloats}

\usepackage{amssymb}

\DeclareMathOperator{\diag}{diag}

\DeclareMathOperator{\mst}{s.t.}

\DeclareMathOperator{\var}{Var}

\DeclareMathOperator{\cov}{Cov}

\renewcommand{\algorithmicrequire}{\textbf{Input:}}

\renewcommand{\algorithmicensure}{\textbf{Output:}}



\newcommand{\argmin}{\arg\!\min}

\newcommand{\argmax}{\arg\!\max}

\newcommand{\Amat}{{\boldsymbol A}}

\newcommand{\Bmat}{{\boldsymbol B}}

\newcommand{\Cmat}{{\boldsymbol C}}

\newcommand{\Dmat}{{\boldsymbol D}}

\newcommand{\Emat}[0]{{{\boldsymbol E}}}

\newcommand{\Fmat}[0]{{{\boldsymbol F}}}

\newcommand{\Gmat}[0]{{{\boldsymbol G}}}

\newcommand{\Hmat}[0]{{{\boldsymbol H}}}

\newcommand{\Imat}{{\boldsymbol I}}

\newcommand{\Jmat}[0]{{{\boldsymbol J}}}

\newcommand{\Kmat}[0]{{{\boldsymbol K}}}

\newcommand{\Lmat}[0]{{{\boldsymbol L}}}

\newcommand{\Mmat}[0]{{{\boldsymbol M}}}

\newcommand{\Nmat}[0]{{{\boldsymbol N}}}

\newcommand{\Omat}[0]{{{\boldsymbol O}}}

\newcommand{\Pmat}[0]{{{\boldsymbol P}}}

\newcommand{\Qmat}[0]{{{\boldsymbol Q}}}

\newcommand{\Rmat}[0]{{{\boldsymbol R}}}

\newcommand{\Smat}[0]{{{\boldsymbol S}}}

\newcommand{\Tmat}[0]{{{\boldsymbol T}}}

\newcommand{\Umat}{{{\boldsymbol U}}}

\newcommand{\Vmat}[0]{{{\boldsymbol V}}}

\newcommand{\Wmat}[0]{{{\boldsymbol W}}}

\newcommand{\Xmat}{{\boldsymbol X}}

\newcommand{\Ymat}[0]{{{\boldsymbol Y}}}

\newcommand{\Zmat}{{\boldsymbol Z}}



\newcommand{\av}{\boldsymbol{a}}

\newcommand{\Av}{\boldsymbol{A}}

\newcommand{\Cv}{\boldsymbol{C}}

\newcommand{\bv}{\boldsymbol{b}}

\newcommand{\cv}{{\boldsymbol{c}}}

\newcommand{\dv}{\boldsymbol{d}}

\newcommand{\ev}[0]{{\boldsymbol{e}}}

\newcommand{\fv}{\boldsymbol{f}}

\newcommand{\Fv}[0]{{\boldsymbol{F}}}

\newcommand{\gv}[0]{{\boldsymbol{g}}}

\newcommand{\hv}[0]{{\boldsymbol{h}}}

\newcommand{\iv}[0]{{\boldsymbol{i}}}

\newcommand{\jv}[0]{{\boldsymbol{j}}}

\newcommand{\kv}[0]{{\boldsymbol{k}}}

\newcommand{\lv}[0]{{\boldsymbol{l}}}

\newcommand{\mv}[0]{{\boldsymbol{m}}}

\newcommand{\nv}{\boldsymbol{n}}

\newcommand{\ov}[0]{{\boldsymbol{o}}}

\newcommand{\pv}[0]{{\boldsymbol{p}}}

\newcommand{\qv}[0]{{\boldsymbol{q}}}

\newcommand{\rv}[0]{{\boldsymbol{r}}}

\newcommand{\sv}[0]{{\boldsymbol{s}}}

\newcommand{\tv}[0]{{\boldsymbol{t}}}

\newcommand{\uv}[0]{{\boldsymbol{u}}}

\newcommand{\vv}{\boldsymbol{v}}

\newcommand{\wv}{\boldsymbol{w}}

\newcommand{\Wv}{\boldsymbol{W}}

\newcommand{\xv}{\boldsymbol{x}}

\newcommand{\yv}{\boldsymbol{y}}

\newcommand{\Xv}{\boldsymbol{X}}

\newcommand{\Yv}{\boldsymbol{Y}}

\newcommand{\zv}{\boldsymbol{z}}



\newcommand{\xvf}{\widetilde{\xv}}

\newcommand{\Fmb}{\boldsymbol{\mathcal{F}}}



\newcommand{\Gammamat}[0]{{\boldsymbol{\Gamma}}}

\newcommand{\Deltamat}[0]{{\boldsymbol{\Delta}}}

\newcommand{\Thetamat}{\boldsymbol{\Theta}}

\newcommand{\Lambdamat}{{\boldsymbol{\Lambda}}}

\newcommand{\Ximat}[0]{{\boldsymbol{\Xi}}}

\newcommand{\Pimat}[0]{{\boldsymbol{\Pi}} }

\newcommand{\Sigmamat}{\boldsymbol{\Sigma}}

\newcommand{\Upsilonmat}[0]{{\boldsymbol{\Upsilon}} }

\newcommand{\Phimat}{\boldsymbol{\Phi}}

\newcommand{\Psimat}{\boldsymbol{\Psi}}

\newcommand{\Omegamat}{{\boldsymbol{\Omega}}}



\newcommand{\Lambdav}{\bm{\Lambda}}

\newcommand{\alphav}{\boldsymbol{\alpha}}

\newcommand{\betav}[0]{{\boldsymbol{\beta}} }

\newcommand{\gammav}{{\boldsymbol{\gamma}}}

\newcommand{\deltav}[0]{{\boldsymbol{\delta}} }

\newcommand{\epsilonv}{\boldsymbol{\epsilon}}

\newcommand{\zetav}[0]{{\boldsymbol{\zeta}} }

\newcommand{\etav}[0]{{\boldsymbol{\eta}} }

\newcommand{\thetav}{\boldsymbol{\theta}}

\newcommand{\iotav}[0]{{\boldsymbol{\iota}} }

\newcommand{\kappav}{{\boldsymbol{\kappa}}}

\newcommand{\lambdav}[0]{{\boldsymbol{\lambda}} }

\newcommand{\muv}{\boldsymbol{\mu}}

\newcommand{\nuv}{{\boldsymbol{\nu}}}

\newcommand{\xiv}{{\boldsymbol{\xi}}}

\newcommand{\omicronv}[0]{{\boldsymbol{\omicron}} }

\newcommand{\piv}{\boldsymbol{\pi}}

\newcommand{\rhov}[0]{{\boldsymbol{\rho}} }

\newcommand{\sigmav}[0]{{\boldsymbol{\sigma}} }

\newcommand{\tauv}[0]{{\boldsymbol{\tau}} }

\newcommand{\upsilonv}[0]{{\boldsymbol{\upsilon}} }

\newcommand{\phiv}{\boldsymbol{\phi}}

\newcommand{\chiv}[0]{{\boldsymbol{\chi}} }

\newcommand{\psiv}{\boldsymbol{\psi}}

\newcommand{\omegav}[0]{{\boldsymbol{\omega}} }



\newcommand{\tsp}{^{\mathsf{T}}}

\newcommand{\inv}{^{-1}}

\newcommand{\ie}{{\em i.e.}}

\newcommand{\wrt}{{\em w.r.t.\,}}

\begin{document}



$\;$\hfill Due: 2019/4/14



\begin{center}

{\LARGE\bf 第二次作业}

\end{center}





\begin{hw}

PCA Algorithm.



\begin{sol}

	对输入的数据X做中心化,得到$\bar{\Xmat}$。记$\Smat=(\Xmat-\bar{\Xmat})^T(\Xmat-\bar{\Xmat})$,则求解第一个主成分$\wv$的问题可以表示为

	\begin{equation}

	\label{eq:PCA_w}

	\begin{aligned}

		&\argmax_{\wv} &\wv^T\Smat\wv\\

		&\mst &\wv^T\wv=1

	\end{aligned}

	\end{equation}

	

	下面描述两种方法求解该主成分$\wv$。

	\begin{enumerate}

		\item[(1)]{

			直接构造拉格朗日函数求解问题~\eqref{eq:PCA_w},求导后可以得到原问题的解在等式

			\begin{equation}

				\label{eq:ei}

				\Smat\wv = \lambda \wv

			\end{equation}

			的解中,即原问题~\eqref{eq:PCA_w}的解$\Smat$的特征向量中。同时注意到在原目标函数中代入等式~\eqref{eq:ei}和$\wv^T\wv=1$,得到优化目标其实为找到最大的$\lambda$,即$\Smat$的最大特征值。

			于是第一种方法可以总结为对$\Smat$做特征值分解,并找到最大的特征值求出它对应的特征向量。

			

			该方法中,对输入数据做中心化的时间复杂度是$\mathcal{O}(Nn)$,求解$\Smat$的时间复杂度是$\mathcal{O}(Nn^2)$,之后对$n\times n$的矩阵$\Smat$做特征值分解的时间复杂度是$\mathcal{O}(n^3)$。综上,该方法的时间复杂度为$\mathcal{O}(Nn^2)$。

		}

		\item[(2)]{

			第二种方法为直接对$\bar{\Xmat}$做奇异值分解(SVD)。注意到从最大投影角度看待PCA时,第一个投影方向$\wv$是能够将数据投影后的分量最大化的方向,这与SVD分解的含义是相吻合的。

			

			该方法中,对输入数据做中心化的时间复杂度是$\mathcal{O}(Nn)$,SVD分解的时间复杂度是$\mathcal{O}(Nn^2)$。综上,该方法的时间复杂度为$\mathcal{O}(Nn^2)$。

		}

	\end{enumerate}

\end{sol}

\end{hw}



\begin{hw}

	Factor Analysis(FA) 

	

	\begin{sol}

		题干中已经给出了先验$P(\yv)$和似然$P(\xv|\yv)$,需要计算后验$P(\yv|\xv)$,其中$\yv$是隐变量,由贝叶斯公式有

		\begin{equation}

			\label{eq:FA_Bay}

			P(\yv|\xv) = \frac{P(\xv|\yv)P(\yv)}{P(\xv)}。

		\end{equation}

		下面计算$P(\xv)$。

		

		因为$\xv$是高斯分布的线性组合,仍然服从高斯分布。计算其期望值

		\begin{equation}

			\label{eq:FA_E}

			E[\xv] = E[\Amat\yv+\muv+\ev] = E[\Amat\yv]+E[\muv]+E[\ev]=\muv。

		\end{equation}

		计算其方差为

		\begin{equation}

		\label{eq:FA_Var}

		\var[\xv] = \var[\Amat\yv+\muv+\ev] = \var[\Amat\yv]+\var[\muv]+\var[\ev]=\Amat\Sigmamat_y\Amat^T+\Sigmamat_e。

		\end{equation}

		故$\xv\sim G(\muv,\Amat\Sigmamat_y\Amat^T+\Sigmamat_e)$,综合已知的先验和似然,后验分布可以用式子~\eqref{eq:FA_Bay}表示。

		

		如果需要整理出$P(\yv|\xv)$的具体公式,可以依据以下公式进行计算:当已知一个联合分布

		\begin{equation}

			\label{eq:FA_uni}

			\left(

				\begin{aligned}

					\xv_a \\

					\xv_b

				\end{aligned}

			\right)

			,\xv\sim G

			\left(

				\left[	

					\begin{aligned}

					\muv_a \\

					\muv_b

					\end{aligned}

				\right],

				\left[

					\begin{aligned}

					&\Sigmamat_{aa} & \Sigmamat_{ab} \\

					&\Sigmamat_{ba} & \Sigmamat_{bb}

					\end{aligned}

				\right]

			\right)

		\end{equation}

		时,可以设一个随机变量$\zv=\xv_b-\Sigmamat_{ba}\Sigmamat_{aa}^{-1}\xv_a$,易知$\zv$也服从高斯分布且均值为$\muv_b-\Sigmamat_{ba}\Sigmamat_{aa}^{-1}\muv_a$,方差为$\Sigmamat_{aa}$的舒尔补$\Sigmamat_{bb}-\Sigmamat_{ba}\Sigmamat_{aa}^{-1}\Sigmamat_{ab}$(这里有$\Sigmamat_{ab}=\Sigmamat_{ba}$)。

		

		那么就有$\xv_b=\zv+\Sigmamat_{ba}\Sigmamat_{aa}^{-1}\xv_a$,按照这个等式关系再计算$P(\xv_b|\xv_a)$,它服从一个高斯分布,均值为$E[\xv_b|\xv_a]=\muv_b+\Sigmamat_{ba}\Sigmamat_{aa}^{-1}(\xv_a-\muv_a)$,方差为$\var[\xv_b|\xv_a]=\var[\zv]=\Sigmamat_{bb}-\Sigmamat_{ba}\Sigmamat_{bb}^{-1}\Sigmamat_{ab}$。因此,有结论

		\begin{equation}

			\label{eq:conclusion}

			P(\xv_b|\xv_a)=G(\muv_b+\Sigmamat_{ba}\Sigmamat_{aa}^{-1}(\xv_a-\muv_a),\Sigmamat_{bb}-\Sigmamat_{ba}\Sigmamat_{aa}^{-1}\Sigmamat_{ab})。

		\end{equation}

		

		求解$P(\yv|\xv)$,要套用上面的公式\eqref{eq:conclusion},可以先计算出$\xv$与$\yv$的协方差:

		\begin{equation}

			\label{eq:FA_cov}

			\begin{aligned}

				\cov(\xv,\yv)&=E[(\xv-\muv)(\yv-0)^T]=E[\xv\yv^T-\muv\yv^T] \\

							 &=E[\Amat\yv\yv^T+\ev\yv^T]=E[\Amat\yv\yv^T]+E[\ev\yv^T] \\

							 &=\Amat E[\yv\yv^T] = \Amat E[(\yv-0)(\yv-0)^T] \\

							 &=\Amat \var[\yv] = \Amat\Sigmamat_y

			\end{aligned}

		\end{equation}

		

		那么可以构造$\xv$与$\yv$的联合分布

		\begin{equation}

			\label{eq:FA_build}

			\left(

				\begin{aligned}

					\xv \\ 

					\yv

				\end{aligned}

			\right)\sim G

			\left(

				\left[	

					\begin{aligned}

						\muv \\

						0

					\end{aligned}

				\right],

				\left[

					\begin{aligned}

						 &\Amat\Sigmamat_y\Amat^T+\Sigmamat_e & \Amat\Sigmamat_y \\

						 &\Amat\Sigmamat_y & \Sigmamat_y

					\end{aligned}

				\right]

			\right)。      

		\end{equation}

		套用之前的公式\eqref{eq:conclusion},得出$P(\yv|\xv)=G(\Amat\Sigmamat_y(\Amat\Sigmamat_y\Amat^T+\Sigmamat_e)^{-1}(\xv-\muv),\Sigmamat_y-\Amat\Sigmamat_y(\Amat\Sigmamat_y\Amat^T+\Sigmamat_e)^{-1}\Amat\Sigmamat_y)$

		

	\end{sol}

\end{hw}



\begin{hw}

	Independent Component Analysis(ICA)

		

	\begin{sol}

		首先我们知道ICA的目标是从混合信号$\xv=\Amat\sv$中,找到一个矩阵$\Wmat$,近似得恢复出$\sv=\Wmat\xv$。我们最大化non-Gaussianity是出于两种考虑:首先,在$\sv$中,最多只有一个分量可以是服从高斯分布的,否则,因为高斯分布在旋转一定角度后可以是相同的,特别是方差为对角矩阵的情况下,这会导致我们无法唯一得恢复出$\sv$,因此要尽可能使得$\sv$的分布不服从高斯分布;其次,由中心极限定理,多个分布线性组合后的分布将比原始的多个分布更接近高斯分布,而在求解$\sv$时,如果不加控制,由于$\sv=\Wmat\xv$其实是$\xv$的线性组合,会导致求出的$\sv$不服从高斯分布假设相悖,于是要人为地最大化non-Gaussianity。

	\end{sol}

\end{hw}



\begin{hw}

	Dimensionality Reduction by FA

	

	\begin{sol}

		首先直接使用了scikit-learn库的因子分析模型,进行了模型选择,这里需要注意它的score()函数虽然可以返回似然值,但是是对所有数据平均后的似然值,为了计算AIC和BIC,我们需要将它再乘以数据量(N)。AIC和BIC中的free parameter的数量,这里使用的是题干中的$(mn+1)$,即只有$\sigma$和矩阵$\Amat$是自由变量。

		\begin{figure}

			\centering

			\includegraphics[width=0.4\linewidth]{Figs/EM_100_3.png}

			\vspace{-0.1cm}

			\caption{直接使用scikit-learn库后的运行结果。}

			\label{fig:EM}

		\end{figure}

		

		另外,自行实现了因子分析的EM算法,同样进行了model selection,但是结果远差于scikit-learn的模型。虽然EM算法最终收敛,同时似然在逐渐上升,但是对每个$m$值计算得到的对数似然值都极低,可能是实现上仍有问题。

		\begin{figure}

			\centering

			\includegraphics[width=0.4\linewidth]{Figs/my_EM.png}

			\vspace{-0.1cm}

			\caption{自行实现EM算法后的运行结果。}

			\label{fig:my_EM}

		\end{figure}

		

		之后,为了测试不同参数对model selection结果的影响,我设置了多组参数($n\in{10,100}$,$m\in{3,7,10}$,$\sigma\in{0.1,1,10}$)来生成数据。Model selection需要根据AIC或BIC的标准,从真实$m$值的左右共五个值中选出一个$m$。我对每个参数组合进行了$1000$次model selection,并记录这1000次model selection中成功选出真实的$m$值的频率。结果如图~\ref{fig:compare}。

		

		可以看到,当$n=10$,即观测数据有$10$维时,只有真实的隐变量y的维度为$3$且$\sigma$为0.1时,可以成功选出的,同时是百分百会成功选出,反之则完全无法正确选出真实$m$值。而当$n=100$时,更多的参数组合可以成功选出真实的$m$值,即使是非常大的$\sigma$也不会影响到最大似然部分的估计。

		

		在实验中,发现使用AIC,BIC的效果相同,推测可能是数据量的原因,这里都是设置了$N=100$。同时一个比较奇怪的现象是我对每组参数随机地做了1000次实验,结果却是十分稳定的,每组参数的成功选出真实$m$的频率稳定不是1就是0,这有待后续观察。

		

		\begin{figure}

			\centering

			\includegraphics[width=0.6\linewidth]{Figs/model_selection.png}

			\vspace{-0.1cm}

			\caption{不同参数设置。}

			\label{fig:compare}

		\end{figure}

	\end{sol}

\end{hw}



\begin{hw}

	Spectral clustering



	\begin{sol}

		为了测试谱聚类我们生成了在不同维度下不同分布的多组数据,并将谱聚类与高斯混合模型(GMM)进行了比较。

		

		从图~\ref{fig:spec_2d}中可以看出,对于前三个环形分布的数据,如果假设同一类的数据是在一个环形上,那么谱聚类的效果远远好于GMM,它可以捕捉到数据分布的连通性信息。而对于由高斯分布生成的数据,高斯混合模型和谱聚类表现相近,但是在第四组数据中可以看出,谱聚类还是很容易将数据误分的,因为它没有能利用到数据生成过程中的高斯分布的先验。而对于最下方的完全随机生成的均匀分布的数据,由于谱聚类和GMM均需要提供类别数量,因而都容易将单一数据划分到多个种类去。

		

		从图~\ref{fig:spec_3d}中的第一个瑞士卷形状的数据分布可以看出,即使是在三维空间中,谱聚类也可以较好地捕捉到一个低维流形,如果数据是沿着瑞士卷分布的,那么会取得很好的聚类效果,同时也许可以使用谱聚类的思路去探索流形嵌入的方法。而第二到第四幅图告诉我们,对三维空间中的高斯分布谱聚类也能取得较好的结果。

		

		更高维的分布不便于可视化就没有再进行比较。总体看来,谱聚类似乎是一个更加通用的聚类方法,对于一些特殊的分布也可以取得较好的聚类结果。但是它不能利用数据生成时的先验信息。

		

		\begin{figure}

			\centering

			\vspace{-0.3cm}

			\includegraphics[width=0.6\linewidth]{./Figs/spec_2d.png}

			\caption{2D平面中的聚类效果。}

			\label{fig:spec_2d}

		\end{figure}

		\begin{figure}

			\centering

			\vspace{-0.3cm}

			\includegraphics[width=0.7\linewidth]{./Figs/spec_3d.png}

			\caption{3D空间中的聚类效果。}

			\label{fig:spec_3d}

		\end{figure}

	\end{sol}	

\end{hw}
\end{document}